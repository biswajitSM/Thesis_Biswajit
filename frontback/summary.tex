\chapter*{Summary}
\label{ch:Summary}
\markboth{Summary}{}
\addcontentsline{toc}{chapter}{Summary}

%\paragraph*{General motivation: 1 page}
Fluorescence is the phenomenon of emission of light at a longer wavelength upon absorption of light.
The high selectivity of fluorescence emission and negligible background enables the detection of single molecules even in a surrounding of billions of other host molecules.
Single molecules provide a detailed kinetics and statistical distribution that is blurred in the  average properties in ensemble measurement.
Asynchronous reactions and full pathways in real time could be studied with single molecule investigation.
In the last two decades, single-molecule technique has revealed detailed biomolecular dynamics: enzyme kinetics, protein energy landscape, protein folding kinetics, strength of chemical bonds, rare and transient events, single molecule stoichiometry.
The single step blinking in the fluorescence of single molecules have been used to locate the position of the molecules with nanometer accuracy leading to a new technique called superresolution miocroscopy.
Fine structures of the organelles in a living cell are imaged with the help of superresoltuon microscopy.
Single-molecule studies has not only been extensively used in biology, it has also found its importance in physics and chemistry.


However molecules with high fluorescence quantum yield could only be detected at single molecule level.
Major fraction of absorbing molecules occurring in nature like proteins, metal complexes show little fluorescence while dissipating rest of the absorbed light in non-radiative channels.
Optical nano-antennas in the form of metallic nanostructures with various shapes are shown to enhance the fluorescence of weak emitters.
The nano-antennas exhibit a resonant mode called surface plasmon resonance (SPR) due to collective motion of free electrons on the metal surface.
Similar to the radio antennas in the macro scale, optical nano-antennas focus electromagnetic energy to much smaller volumes with intensity few orders of magnitude higher than the incident beam.
The high intensity in a volume of a few attoliters helps selectively excite the probe molecules with maximum intenisty.
The radiative rates of the emitters are also amplified by the high local density of states of the photons near nano-antenna.
Enhancement of both excitation and emission rates can lead to fluorescence enhancement by few orders of magnitude enabling single moleculde detection of weak emitters.


In this thesis, fluorescence enhancement by gold nanorod is exploited to improve the detection of single molecules at high concentrations and with high background.
We present methods to control single molecules near the nanorod both in space and time.
Furthermore, the fluorescence enhancement is characterized down to single photon level.
With the future prospect to explore biomolecule with optical nanoantennas, electron transfer in azurin is investigated in detail.
While characterizing the rates of electron transfer of azurin, we embark on the dynamic behavior of the rates which is related to the structure of proteins.
Dynamical heterogeneity, fundamental to the functioning of proteins is discussed in detail in the final chapter.

%\paragraph*{Chapter-2}
Fluorescence correlation spectroscopy (FCS) measures the fluctuation of signal while molecules diffuse in a fluid.
Diffusional characteristics and dynamic behavior of bio- molecules can be extracted from the correlation curves obtained in FCS.
However to provide the necessary intensity fluctuation in FCS measurements, the concentrations of fluorescent molecules need to be in the pico- to nano-molar range in the diffraction limited optical volume of femtoliters.
But a majority of physiological reactions involving proteins and enzymes occur in micromolar to millimolar concentrations. 
\textbf{Chapter 2} details the use of gold nanorod as optical nanoantenna to overcome the diffraction limit of light and perform FCS at a micromolar concentration.
ATTO647N with a quantum yield of \SI{70}{\percent} was internalized to a lipid bilayer supported on a glass substrate.
The free mobility of ATTO647N in the bilayer gave access to the hotspots of the nanorods.
A maximum fluorescence enhancement by a factor of five was obtained, though the nanorod could enhance up to two orders of magnitude for such a dye with high quantum yield.
The limited enhancement was attributed to the inaccessibility of ATTO647N to the hottest (highest electric field) spot around the nanorod due to the confinement of the dye in the bilayer at the bottom of the nanorod.
The intensity correlation resulted in a two component decay corresponding to the diffusion in the laser focus and in the near-field of the nanorod.
The near-field volume was estimated from the diffusion time and the diffusion coefficient in the bilayer.
The low enhancement factor and high background from the unenhanced fluorophores lowered the contrast of the correlation corresponding to the near-field.
However the lifetime of the enhanced signal was much shorter than that of unenhanced fluorescence which could be used to discriminate the enhanced signal from the unenhanced fluorescence.
The correlation of selected signal based on fluorescence lifetime made a significant improvement in the contrast of the correlation.
Higher the contrast of correlation higher the concentration of molecules could be used for FCS measurements.
Azurin labeled with ATTO655 was anchored onto the bilayer to show the flexibility of the method to biochemical applications.
Diffusion times corresponding to both far-field and near-field slowed down by a factor of five due to the bulky anchor in the bilayer.
The biggest advantage of employing a bilayer in the enhacement experiment was the minimal non-specific interaction of the protein (or the dye) with the substrate and probably with the nanostructure as well.


%\paragraph*{Chapter-3}
Freely diffusing dyes in the bilayer around gold nanorod led to exponential-like distribution of enhancement factors due to the strong inhomogeneity in the electric field around gold nanorod.
The emission rate and quenching of fluorophores also depend strongly on the position and orientation of emitter with respect to the nanorod.
Such variation in excitation and emission processes not only result in unwanted heterogeneity but also can alter the photochemistry of the probe.
\textbf{Chapter 3} explains a method based on transient binding by DNA to observe enhanced fluorescence from the same spot at the tip of a gold nanorod.
Few docking DNAs was chemically bound to the surface of the nanorod while rest of the nanorod surface was passivated with polyethylene glycol (PEG).
Complimentary DNAs (10 base pairs) with Cy5 dye known as imager was flushed around the nanorod.
An imager strand temporarily binds to the docking strand giving intensity bursts with a duration of around few seconds.
The average number of docking strands leading to fluorescence enhancement was controlled to one per nanorod by adjusting the ratio between the docking strands and the passivating ligands.
As an imager strand is replaced by a fresh imager strand, any number of single molecules with the same enhancement factor can be recorded.
The position of the docking DNA was not controlled to be tip specific.
Variation in the position of the docking strand led to variation of enhancement factor among different nanorods with an exponential distribution with an average enhancement factor of 25.
But once a nanorod is identified with a satisfactory enhancement factor, the nanorod can be characterized in detail and single-molecule measurements can be performed over a long time.


%\paragraph*{Chapter-4}
Fluorescence enhancement is usually estimated from binned time traces with a certain integration time.
The burst with highest intensity is selected ('cherry-picking') to estimate the enhancement factor.
Here the integration time is an arbitrary parameters and there is no warranty that different binning time wouldn't lead to different enhancement factor.
Distribution of time delays between two consecutive photons (interphoton times) has been used in \textbf{Chapter 4} to characterize emission of emitters and extract information without introducing any arbitrary parameter.
We present a model to relate interphoton delay distribution to the spatial intensity distribution of diffusing molecules in the limit of slow diffusion.
The model is first successfully tested with the simple case of an emitter switching between two intensity levels as obtained in transient binding experiemnt.
The interphoton delay histogram for the diffusing dyes in the bilayer agrees with the model in the high concentration limit, but deviates at low concentrations.
The deviation at low concentration could be due to the non-negligible contribution from the false events in the detector or photobleaching of the dye.
The model for the nanorod case was developed with near-field component taken as a steeply decaying power law of distribution.
Enhancement factors estimated from the model was consistent with the binning method with a binning time of \SI{100}{\us} but with the additional advantage of not requiring to choose the right binning time.


%\paragraph*{Chapter-5}
The next goal of the project was to combine the fluorescence enhancement by a gold nanorod with a redox active azurin labeled with a dye.
The functionalized gold nanorod could then be used as a sensor to measure redox potential inside a living cell and the fluorescence enhancement would provide high contrast to the signal against autofluorescence background of the cell.
However we immediately realized that the electron transfer properties of azurin by single molecule fluorescence is not well characterized.
\textbf{Chapter 4} describes the electron transfer properties of azurin \textit{in vitro} where the solution potential is controlled electrochemically.
Cu azurin labeled with ATTO655 (FRET donor) showed distinct fluorescence intensity in the oxidized and reduced state of copper.
The duration of azurin in reduced and oxidized  state was determined by the bright and dark times in the fluorescence time trace.
Midpoint potential of single azurin molecules were calculated from the ratio of bright and dark times at a given solution potential using Nernst equation.
The distribution of midpoint potentials was centered at \SI{5}{\mV} with full width half maximum of \SI{35}{\mV} which was the narrowest among the reported literatures.
Such a narrow distribution in the midpoint potentials obtained due the minimal interaction of azurins with the passivating substrate.
But what is reason for the distribution over \SI{35}{\mV}?
Is the distribution due to subpopulations of azurin with different redox properties or due to dynamical behavior of azurin?
To answer this question we investigated single azurins over a longer period of time.
The bright and dark times showed a stretched exponential distribution indicating the variation in electron transfer rates.
Such variation in rates was attributed to the different conformations of azurin structure known as conformational substates.
Correlation of bright and dark times showed a characteristic decay time of around tens of seconds indicating the period for which azurin spends in each of the conformational substate.
As the number of substates could be large, no two azurins were found to have identical correlations.