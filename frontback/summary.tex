\chapter*{Summary}
\label{ch:Summary}
\markboth{Summary}{}
\addcontentsline{toc}{chapter}{Summary}

%\paragraph*{General motivation: 1 page}
Absorption of light at a particular wavelength followed by emission of light at longer wavelengths is called fluorescence.
% When a sample is illuminated by light of a given wavelength, it often emits light at longer wavelengths. This phenomenon is called fluorescence.
Fluorescence as a monitoring tool is highly selective and presents negligible background for properly purified samples. These features enable the detection of single molecules, even when they are surrounded by billions of other, non-fluorescent host molecules.
Detailed kinetics and statistical distributions, which are blurred in the averaged ensemble measurements, become directly accessible in single-molecule measurements.
Chemical reactions that cannot be synchronized (asynchronous reactions) and the associated pathways can be studied in real time with single molecules.
In the last two decades, single molecules have revealed detailed biomolecular dynamical processes such as enzyme kinetics, protein energy landscape, protein folding kinetics, breaking of chemical bonds, rare and transient events, and more.
Digital blinking in the fluorescence of single molecules has been used to locate molecules with nanometer accuracy, giving rise to a new microscopy technique called superresolution microscopy.
Fine structures of the organelles in a living cell are imaged with the help of superresolution microscopy.
Single-molecule studies have not only been extensively used in biology, they have also found numerous applications in physics and chemistry.
However, the only molecules used so far in single-molecule experiments were molecules with high fluorescence quantum yield.
The overwhelming majority of absorbing molecules occurring in nature, such as proteins or metal complexes show only weak fluorescence, as they dissipate much of the absorbed energy into non-radiative channels.
In the past 20 years, it was realized that optical nano-antennas in the form of metallic nanostructures with various shapes can enhance the fluorescence of weak emitters.
These nano-antennas often exhibit a resonant mode called surface plasmon resonance (SPR), due to collective motion of free electrons in the metal.
Similar to radio antennas at macroscopic scales, optical nano-antennas focus electromagnetic energy to much smaller sizes than the electromagnetic wavelength, with associated intensities which are several orders of magnitude higher than those of the incident waves.
High local intensities in a volume of a few attoliters help selectively excite a handful of probe molecules with negligible contribution from the other probe molecules in the diffraction-limited volume.
The radiative rates of the emitters can also be amplified by the high local density of photon states in the near field of a nano-antenna.
The enhancements of both excitation and emission rates can lead to fluorescence enhancement by a few orders of magnitude, thereby enabling the detection of weak emitters at the single-molecule level.


In this thesis, fluorescence enhancement by gold nanorods is shown to improve the detection of single molecules at high concentrations and with high background.
We demonstrate methods to bind single molecules transiently near a nanorod.
Furthermore, we present novel methods of data treatment to characterize fluorescence enhancement from the stream of measured single photons, without any need for arbitrary bins or thresholds.
With the future prospect of tracking and monitoring biomolecules with optical nanoantennas, we started a detailed investigation of electron transfer in the metalloprotein copper-azurin.
While characterizing the rates of electron transfer of azurin under controlled electrochemical potential, we realized that single proteins show dynamical heterogeneity, i.e., fluctuations of their reaction rates, which we attribute to transitions between conformational substates of the protein.
This process is deemed fundamental for the function of proteins and is discussed in the final chapter.


%\paragraph*{Chapter-2}
Fluorescence correlation spectroscopy (FCS) measures signal fluctuations due to diffusion of fluorescent molecules in a fluid environment.
Diffusion characteristics and dynamics of bio- molecules can be extracted from the correlation functions obtained in FCS.
However, to provide the necessary intensity fluctuations in conventional FCS measurements with a diffraction-limited optical volume of femtoliters, the concentration of fluorescent molecules needs to be in the pico- to nano-molar range, whereas a majority of physiological reactions involving proteins and enzymes occur in micromolar to millimolar concentrations. 
\textbf{Chapter 2} details the use of a gold nanorod as an optical nanoantenna to overcome the diffraction limit of light and perform FCS at micromolar concentrations.
The dye ATTO647N with a quantum yield of \SI{70}{\percent} was included into a lipid bilayer supported on a glass substrate.
As ATTO647N was freely diffusing in the bilayer, it could access the hot spots of the nanorods.
A maximum fluorescence enhancement of five was observed, although calculations indicate that the nanorod could enhance the fluorescence of this dye by up to two orders of magnitude at the optimum location within the hot spot.
We attribute the fairly low enhancement to confinement of the dye in the bilayer, which prevents access to the hottest (highest electric field) spot close to the nanorod tip.
The intensity correlation presented a two-component decay corresponding to diffusion in the laser focus with low fluorescence intensity and diffusion in the near-field of the nanorod with high fluorescence intensity.
The near-field volume was estimated from the diffusion time and the diffusion coefficient in the bilayer.
The low intensity and high background from the unenhanced fluorophores lower the contrast of the correlation component corresponding to the near-field.
However, the lifetime of the enhanced signal is much shorter than that of the unenhanced signal. We used this property to discriminate the enhanced signal from the unenhanced one.
The correlation of photons selected through their short delay from excitation significantly improves the contrast of the near-field correlation.
The higher the correlation contrast, the higher is the concentration of molecules that can be used in FCS measurements.
As a test, azurin labeled with ATTO655 was anchored onto the bilayer to show the application of the method to biochemical samples.
Diffusion in both the far field and the near field was slowed down by a factor of five due to the bulky anchor of the protein in the bilayer.
The biggest advantage of employing a bilayer in the enhancement experiment was the complete elimination of non-specific interactions between the labelled protein and the substrate.
We assume that interactions with the nanostructure were removed too.


%\paragraph*{Chapter-3}
Freely diffusing dyes in the bilayer around a gold nanorod lead to an exponential-like distribution of enhancement factors due to the strong inhomogeneity in the electric near field.
The emission and quenching rates of fluorophores also depend strongly on their position and orientation with respect to the nanorod.
Such variation in excitation and emission rates not only result in unwanted heterogeneity, but can also alter the photochemistry of the fluorophores.
\textbf{Chapter 3} demonstrates a method based on transient binding by DNA to reduce this heterogeneity.
By observing enhanced fluorescence from the same spot at the tip of the gold nanorod, we can study many single fluorophores with exactly the same excitation and emission rates.
A small number of single DNA strands (docking strands) were chemically bound to the surface of the nanorod while the rest of the nanorod surface was passivated with polyethylene glycol (PEG).
Complementary single strands of DNA (10 base pairs, imager strands) labeled with Cy5 were injected around the nanorod and left to hybridize with the docking strands.
The imager strands temporarily bound to the docking strands, giving fluorescence intensity bursts with a typical duration of a few seconds.
The average number of docking strands in the enhancement spot was controlled down to one per nanorod by adjusting the ratio between the docking strands and the passivating ligands.
The position of the docking DNA was not controlled to be tip-specific.
Variation in the position of the docking strand led to variation of enhancement factors among different nanorods.
The enhancement factors are exponentially distributed with an average value of 25.
As an imager strand at the docking site eventually unbinds and is replaced by a fresh imager strand, arbitrary numbers of single molecules can be studied with exactly the same enhancement factor.
Once a nanorod with a satisfactory enhancement factor has been found, it can be characterized in detail and used over long times to perform single-molecule measurements.


%\paragraph*{Chapter-4}
Fluorescence enhancement is usually estimated from binned time traces with a certain integration time.
The standard way to estimate the enhancement factor is to select the burst with the highest intensity ('cherry-picking').
However, the integration time is an arbitrary parameter and there is no warranty that different binning times would lead to the same enhancement factor.
In \textbf{Chapter 4}, we propose to use the distribution of time delays between two consecutive photons (or interphoton delay distribution) to characterize the emission bursts and to extract information without introducing any arbitrary parameter.
We relate the interphoton delay distribution theoretically to the spatial intensity distribution sampled by diffusing molecules in the limit of slow diffusion.
Our model is first successfully tested with the simple case of an emitter switching between two intensity levels, such as obtained in transient binding experiments.
The interphoton delay distributions for diffusing dyes in bilayers agree with the model in the limit of high concentrations, but deviate from it at low concentrations.
This deviation could be due to the dark detector counts and/or to photobleaching of the dye.
The model for the nanorod case was developed with a near-field component taken as a steeply decaying power law of distance.
Enhancement factors estimated from the model were consistent with the binning method (binning time \SI{100}{\us}) but with the additional advantage of not requiring to choose the right binning time.


%\paragraph*{Chapter-5}
The next goal of the project was to combine the fluorescence enhancement by a gold nanorod with a redox-active protein (copper-azurin) labeled with a dye.
The functionalized gold nanorod might be used in the future as a sensor to measure redox potentials inside living cells.
Fluorescence enhancement by the rod would improve the contrast against cell autofluorescence, and thereby improve measurement quality.
However, we immediately realized that the electron transfer properties of azurin had not been characterized well enough at the single-molecule level.
\textbf{Chapter 5} reports an \textit{in vitro} study of electron transfer in single azurin molecules, under electrochemically controlled potential.
Copper azurin labeled with ATTO655 as energy donor showed distinct fluorescence intensities in the oxidized and reduced states of the copper.
The lifetimes of reduced and oxidized states of azurins were determined by the bright and dark times in the fluorescence time trace.
The midpoint potentials of single azurin molecules were calculated from the ratio of bright and dark times at a given solution potential, using the Nernst equation.
The distribution of midpoint potentials was centered at \SI{5}{\mV} vs the saturated calomel electrode with a full width at half maximum of \SI{35}{\mV}, which is the smallest width value among those reported in the literature.
Such a narrow distribution of midpoint potentials is consistent with a minimal interaction of azurin with the passivated substrate.
Yet, the origin of the remaining width, and the spatial or temporal nature of this distribution were still unclear.
To answer these questions, we investigated single azurin molecules over longer periods of time.
The bright and dark times in fluorescence time traces showed stretched-exponential distributions, indicating temporal variation in electron-transfer rates.
We attribute such rate variations to the different conformations of azurin known as conformational substates.
Correlation of bright and dark times in our experimentally accessible window showed a characteristic decay time of around tens of seconds, indicating the average time azurin spends in each of its conformational substates.
That no two individual azurin molecules were found to have identical correlations, is consistent with a very large number of conformational substates.