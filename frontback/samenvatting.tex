\chapter*{Samenvatting}
\label{ch:Samenvatting}
\markboth{Samenvatting}{}
\addcontentsline{toc}{chapter}{Samenvatting}

Absorptie van licht door organische materialen bij een specifieke golflengte gevolgd door emissie
van licht op langere golflengten wordt fluorescentie genoemd. Fluorescentie is zeer selectief en
presenteert verwaarloosbare achtergrond voor correct gezuiverde monsters. Deze functies maken het mogelijk de detectie van afzonderlijke moleculen, zelfs als ze worden omringd door miljarden andere, niet-fluorescente gastheermoleculen. Gedetailleerde kinetiek en statistische verdelingen, welke worden gemaskeerd in de gemiddelde ensemble metingen, worden direct toegankelijk in enkele-molecuul metingen. Chemische reacties die niet kunnen worden gesynchroniseerd (asynchrone reacties) en de bijbehorende routes kunnen in real time met enkele moleculen worden bestudeerd. In de laatste twee decennia hebben afzonderlijke moleculen een gedetailleerde biomoleculaire dynamiek onthuld van processen zoals enzymkinetiek, eiwit-energielandschappen, eiwitvouwingskinetiek, het breken van chemische bindingen, zeldzame en transitoire gebeurtenissen, en meer. Digitaal knipperende fluorescentie van enkele moleculen is gebruikt om moleculen te lokaliseren met nanometer nauwkeurigheid, met een nieuw microscopische techniek genaamd superresolutiemicroscopie. Fijne structuren van de organellen in een levende cel worden afgebeeld met behulp van superresolutiemicroscopie. Enkel-moleculaire studies zijn niet alleen uitgebreid gebruikt in de biologie, maar hebben ook talloze toepassingen gevonden in de natuurkunde en scheikunde. Echter, de enige gebruikte moleculen tot dusverre in enkele-molecuul experimenten waren moleculen met een hoog fluorescente kwantumopbrengst. De overweldigende meerderheid van absorberende moleculen die voorkomen in de natuur, zoals eiwitten of metaalcomplexen, vertonen slechts zwakke fluorescentie, omdat ze het merendeel van de geabsorbeerde energie verdrijven via niet-radiatieve kanalen. In de afgelopen 20 jaar zijn optische nano-antennes in de vorm van metalen nanostructuren gerealiseerd, met verschillende vormen die de fluorescentie van zwakke stralers verbeteren. Deze nano-antennes vertonen vaak een resonerende modus genaamd surface plasmon resonance (SPR, oppervlakte plasmonische resonantie), als gevolg van collectieve beweging van vrij elektronen in het metaal. Vergelijkbaar met radioantennes op macroscopische schalen, optische nano-antennes focusseren elektromagnetische energie tot veel kleinere maten dan de elektromagnetische golflengte, met bijbehorende intensiteiten die verschillende ordes van grootte hoger zijn dan die van de inkomende golven. Hoge lokale intensiteiten in een volume van een paar attoliters helpen selectief een handvol sondemoleculen te exciteren met een verwaarloosbare bijdrage van de andere sondemoleculen in het diffractie-gelimiteerde volume. De stralingspercentages van de zenders kunnen ook worden versterkt door de hoge lokale dichtheid van foton-toestanden in het nabije veld van een nano-antenne. De verbeteringen van zowel excitatie- als emissiewaarden kunnen leiden tot verbetering van de fluorescentie met enkele ordes van grootte, waardoor detectie mogelijk wordt van zwakke emitters op het niveau van één molecuul.


In dit proefschrift wordt aangetoond dat fluorescentieverbetering door gouden nano-staafjes de detectie verbetert van enkele moleculen in hoge concentraties en met een hoge achtergrond. Wij demonstreren methoden om enkele moleculen kortstondig in de buurt van een nano-staafje te binden. Voorts presenteren we nieuwe methoden voor data behandeling om fluorescentieverbetering te karakteriseren uit de stroom gemeten afzonderlijke fotonen, zonder de noodzaak voor willekeurige afrondingen of drempels. Met het toekomstperspectief van het volgen en monitoren van biomoleculen met optische nano-antennes begonnen we een gedetailleerd onderzoek naar de elektronenoverdracht in de metallo-eiwit koper-azurine. Tijdens het karakteriseren van de reactiesnelheden van elektronenoverdracht van azurine onder gecontroleerde elektrochemisch potentieel, realiseerden we ons dat enkele eiwitten dynamische heterogeniteit laten zien, d.w.z. fluctuaties van hun reactiesnelheden, die we toeschrijven aan overgangen tussen conformationele subtoestanden van het eiwit. Dit proces wordt geacht fundamenteel te zijn voor de functie van eiwitten en wordt besproken in het laatste hoofdstuk.


Fluorescentie Correlatie Spectroscopie (FCS) meet signaalfluctuaties als gevolg van diffusie van fluorescente moleculen in een vloeibare omgeving. Diffusiekarakteristieken en dynamiek van biomoleculen kan worden afgeleid uit de verkregen correlatiefuncties in FCS. Om echter de noodzakelijke intensiteitsschommelingen in conventionele FCS metingen te verschaffen met een diffractie gelimiteerd optisch volume van femtoliters, moet de concentratie
van fluorescerende moleculen in het pico- tot nanomolaire bereik zijn, terwijl de meerderheid van fysiologische reacties met eiwitten en enzymen voorkomen in micro- tot millimolaire concentraties.  Hoofdstuk 2 beschrijft het gebruik van een gouden nano-staafje als een optisch middel nano-antennes om de diffractielimiet van licht te overwinnen en FCS op micromolair concentratie uit te voeren. De fluorofoor ATTO647N met een kwantumopbrengst van \SI{70}{\percent} werd opgenomen in een dubbele lipide laag ondersteund op een glassubstraat. Omdat ATTO647N vrij diffundeerde in de lipide lagen, kon het toegang krijgen tot de hotspots van de nano-staafjes. Een maximale fluorescentieverbetering van vijf werd waargenomen, hoewel berekeningen aangaven dat het nano-staafje de fluorescentie van deze kleurstof met maximaal twee ordes van grootte zou kunnen verbeteren op de optimale locatie binnen de hot spot. We schrijven de vrij lage verbetering toe aan de opsluiting van de kleurstof in de lipide lagen, die toegang tot de heetste (hoogste elektrische veld) plek dicht bij de tip van het nano-staafje. De correlatie van de intensiteit gaf een corresponderend twee-componenten verval aan naar diffusie in de laserfocus met lage fluorescentie-intensiteit en diffusie in het  elektrische nabije-veld van het nano-staafje met hoge fluorescentie-intensiteit. Het nabije-veld volume was geschat op basis van de diffusietijd en de diffusiecoëfficiënt in de dubbele lipide laag. De lage intensiteit en hoge achtergrond van de niet-verbeterde fluoroforen verlagen het contrast van de correlatiecomponent die overeenkomt met het nabije-veld. De levensduur van het versterkte signaal is veel korter dan dat van het niet-versterkte signaal. We hebben deze eigenschap gebruikt om onderscheid te maken tussen het verbeterde signaal en het niet-versterkte signaal. De correlatie van fotonen geselecteerd door hun korte vertraging van excitatie verbetert aanzienlijk het contrast van de nabije-veld correlatie. Hoe hoger het correlatiecontrast, des te hoger is de concentratie van moleculen die kan worden gebruikt in FCS-metingen. Als test is azurine gelabeld met ATTO655 en verankerd op de lipide laag om de toepassing van deze methode te tonen voor biochemische monsters. Diffusie in zowel het verre veld als het nabije veld werd vertraagd met een factor vijf vanwege het omvangrijke anker van het eiwit in de dubbele laag. Het grootste voordeel van het gebruik van een dubbele laag in het verbeteringsexperiment was de volledige verwijdering van niet-specifieke interacties tussen het gelabelde eiwit en het substraat. We nemen aan dat interacties met de nano-structuur ook zijn verwijderd.


Vrij diffunderende kleurstoffen in de dubbele lipide laag rond een gouden nano-staafje leiden tot een exponentiële vorm van verdeling van de versterkingsfactoren vanwege de sterke discontinuïteit in het elektrische nabij veld. De emissie- en uitdovingssnelheden van fluoroforen zijn ook sterk afhankelijk van hun positie en oriëntatie ten opzichte van het nano-staafje. Een dergelijke variatie in excitatie- en emissiesnelheden resulteert niet alleen in ongewenste heterogeniteit, maar kan ook de fotochemie van de fluoroforen veranderen. Hoofdstuk 3 demonstreert een methode gebaseerd op de kortstondige binding door DNA om deze heterogeniteit te verminderen. Door het observeren van versterkte fluorescentie op dezelfde plek aan het uiteinde van het gouden nano-staafje kunnen we vele enkele fluoroforen bestuderen met
exact dezelfde excitatie- en emissiewaarden. Een klein aantal afzonderlijke DNA-strengen
(koppelingsstrengen) werden chemisch gebonden aan het oppervlak van het nano-staafje terwijl de rest van het oppervlak van het nano-staafje inert werd gemaakt met polyethyleenglycol (PEG). Complementaire enkele strengen DNA (10 basenparen, visualisatie-strengen) gemerkt met Cy5 werden geïnjecteerd rond het nano-staafje om te hybridiseren met de koppelingsstrengen. De visualisatie-strengen bonden tijdelijk aan de koppelingsstrengen waardoor verhoogde intensiteit  van fluorescentie werd verkregen met een typische duur van een paar seconden. Het gemiddelde aantal koppelingsstrengen in de verbetering spot werd tot één per nano-staafje gereguleerd door de verhouding tussen de koppelingsstrengen en de passiverende liganden aan te passen. De positie van het koppelings-DNA was niet gereguleerd om specifiek voor de tip te zijn. Variatie in de positie van de koppelstreng leidde tot variatie van versterkingsfactoren tussen verschillende nano-staafjes. De versterkingsfactoren zijn exponentieel verdeeld met een gemiddelde waarde van 25. Als een visualisatie-streng bij de koppelplek uiteindelijk afkoppelt en wordt vervangen door een nieuw visualisatie-streng, dan worden willekeurige getallen van enkele moleculen met exact dezelfde versterkingsfactor waargenomen. Wanneer een nano-staafje met een bevredigende versterkingsfactor is gevonden, deze kan worden gekarakteriseerd in detail en langdurig gebruikt worden om metingen met één enkel molecuul uit te voeren.


Fluorescentieverbetering wordt vaak geschat op basis van tijdsporen met een bepaalde
integratie tijd. De standaardmanier om de versterkingsfactor te schatten is door de
burst met de hoogste intensiteit te selecteren ('cherry-picking'). De integratietijd is echter een
willekeurige parameter en er is geen garantie dat verschillende tijdschalingen zouden leiden
dezelfde versterkingsfactor. In hoofdstuk 4 stellen we voor de verdeling van te gebruiken
tijdsvertragingen tussen twee opeenvolgende fotonen (of inter-fotonvertragingsverdeling) te karakteriseren naar de emissie-bursts en om informatie te extraheren zonder een willekeurige parameter hiervoor te introduceren. We relateren de inter-fotonvertragingsdistributie theoretisch aan de ruimtelijke intensiteitsverdeling bemonsterd door diffunderende moleculen in de limiet van langzame diffusie. Ons model is succesvol getest met het eenvoudige geval van een zender die schakelt tussen twee intensiteitsniveaus, zoals verkregen in transitoire bindingsexperimenten. De inter-fotonvertragingsverdelingen voor diffunderende fluoroforen in dubbele lipide lagen komen overeen met het model in de limiet van hoge
concentraties, maar wijken daarvan af bij lage concentraties. Deze afwijking kan te wijten zijn aan het tellen van achtergrond fotonen door de detector en / of door fotobleking van de kleurstof. Het model voor een nano-staafje werd ontworpen met een nabije-veldcomponent genomen als een sterk vervallende macht als functie van de afstand. Versterkingsfactoren geschat op basis van het model kwamen overeen met de tijdschalingsmethode (tijdschaling  100 microseconde) maar met het extra voordeel dat het niet nodig is om de juiste tijdschalingstijd te kiezen.


Het volgende doel van het project was om de fluorescentieverbetering te combineren met een goud nano-staafje met een redox-actief eiwit (koper-azurine), gemerkt met een fluorofoor. Het gefunctionaliseerde gouden nano-staafje kan in de toekomst worden gebruikt als een sensor om redox-potentialen te meten in levende cellen. Fluorescentieverbetering door het staafje zou het contrast kunnen verbeteren tegenover cel-autofluorescentie, en daarmee de kwaliteit van de waarneming verbeteren. Echter, wij realiseerden ons onmiddellijk dat de elektronenoverdrachtseigenschappen van azurine niet goed genoeg was gekarakteriseerd op het niveau van één enkel molecuul. Hoofdstuk 4 rapporteert een in vitro studie van elektronenoverdracht in enkele azurine-moleculen, onder een elektrochemisch gecontroleerde potentieel. Koper-azurine gelabeld met ATTO655 als energiedonor vertoonde duidelijke fluorescentie intensiteiten in de geoxideerde en gereduceerde toestand van het koper. De levensduren van de gereduceerde en geoxideerde toestanden van azurines werden bepaald door de heldere en donkere tijden in het fluorescentie tijdspoor. De middelpuntspotentialen van enkele azurine-moleculen werden berekend van de verhouding van heldere en donkere tijden bij een bepaald oplossingspotentieel, met behulp van de Nernst-vergelijking. De verdeling van de middelpuntspotentialen was gecentreerd rond 5mV met een volledige breedte op
half maximum van 35mV, wat de kleinste waarde is van de breedte gerapporteerd in de
literatuur. Zo'n smalle verdeling van middelpuntspotentialen is consistent met een minimum interactie van azurine met het gepassiveerde substraat. Echter, de oorsprong van de resterende breedte, en de ruimtelijke of temporele aard van deze verdeling was nog steeds onduidelijk. Om deze vragen te beantwoorden hebben wij enkele azurine-moleculen gedurende langere tijd onderzocht. De heldere en donkere tijden in fluorescentietijdsporen toonden rek-exponentiële verdelingen, een aanwijzing voor temporele variatie in elektron-overdrachtssnelheden. We schrijven dergelijke variaties in de snelheden toe naar de verschillende toestanden van azurine, bekend als conformationele subtoestanden. Correlatie van de heldere en donkere tijden in ons experimenteel toegankelijke venster toonde een kenmerkende vervaltijd van tientallen seconden, hetgeen de gemiddelde tijd aangeeft die azurine in elke tijd doorbrengt van zijn conformationele subtoestanden. Dat er geen twee individuele azurine-moleculen werden gevonden om identieke correlaties te hebben, is consistent met een zeer groot aantal conformationele subtoestanden.
