\chapter*{Samenvatting}
\label{ch:Samenvatting}
\markboth{Samenvatting}{}
\addcontentsline{toc}{chapter}{Samenvatting}

Absorptie van licht door organische materialen bij een specifieke golflengte gevolgd door emissie
van licht bij langere golflengten wordt fluorescentie genoemd. Fluorescentie is zeer selectief en geeft een verwaarloosbare achtergrond voor goed gezuiverde monsters. Deze eigenschappen maken het mogelijk om afzonderlijke moleculen te detecteren, zelfs als ze worden omringd door miljarden andere, niet-fluorescerende gastheermoleculen. Gedetailleerde kinetiek en statistische verdelingen, welke onzichtbaar zijn in een ensemble meting, worden direct toegankelijk in ’single-molecule’ metingen. Chemische reacties die niet kunnen worden gesynchroniseerd (asynchrone reacties) en de bijbehorende reactiepaden kunnen in real time met enkele moleculen worden bestudeerd. In de laatste twee decennia hebben ’single-molecule’ studies tot in detail de dynamiek laten zien van processen zoals enzymkinetiek, eiwit-energielandschappen, eiwitvouwingskinetiek, het verbreken van chemische bindingen, en meer. Digitaal knipperende fluorescentie van enkele moleculen is gebruikt om moleculen te lokaliseren met nanometer nauwkeurigheid, met een nieuwe techniek genaamd superresolutiemicroscopie. Hoogopgeloste  structuren van de organellen in een levende cel konden worden afgebeeld met behulp van superresolutiemicroscopie. Single-molecule studies zijn niet alleen veel gebruikt in de biologie, maar hebben ook talloze toepassingen gevonden in de natuurkunde en de scheikunde. Echter, de enige tot nog toe gebruikte moleculen voor ’single-molecule’ experimenten waren moleculen met een hoge kwantumopbrengst voor de fluorescentie. De overweldigende meerderheid van absorberende moleculen die voorkomen in de natuur, zoals eiwitten of metaalcomplexen, vertonen slechts zwakke fluorescentie, omdat ze het grootste deel van de geabsorbeerde energie verliezen via donkerprocessen. In de afgelopen 20 jaar zijn optische nano-antennes in de vorm van metalen nanostructuren gerealiseerd, met verschillende vormen die de fluorescentie van zwakke stralers verbeteren. Deze nano-antennes vertonen vaak een resonante mode genaamd ``surface plasmon resonance'' (SPR, oppervlakte plasmonische resonantie), als gevolg van de collectieve beweging van vrije elektronen in het metaal. Vergelijkbaar met radioantennes op macroscopische schaal, focusseren optische nano-antennes de elektromagnetische energie in volumes met veel kleinere afmetingen dan de elektromagnetische golflengte. De daarmee gepaard gaande intensiteiten zijn verschillende ordes van grootte hoger dan die van de inkomende golven. Hoge lokale intensiteiten in een volume van een paar attoliters helpen selectief een handvol moleculen te exciteren met een verwaarloosbare bijdrage van de moleculen in het (veel grotere) diffractie-gelimiteerde volume. De emissie kan ook nog worden versterkt door de hoge lokale dichtheid van toestanden in de nano-antenne. De verbeteringen van zowel excitatie- als emissiewaarden kunnen leiden tot verhoging van de fluorescentie met enkele ordes van grootte, waardoor detectie mogelijk wordt van zwakke emitters op het niveau van één molecuul. 


In dit proefschrift wordt aangetoond dat verhoging van de fluorescentie intensiteit door gouden nano-staafjes de detectie mogelijk maakt van enkele moleculen in oplossingen met een hoge concentratie en met een hoge achtergrond. Wij bespreken methoden om ’single molecules’ kortstondig aan een nano-staafje te binden. Voorts presenteren we nieuwe methoden voor data analyse van een stroom van afzonderlijke fotonen, zonder de noodzaak van afrondingen of drempels. Met de mogelijkheid in het achterhoofd om biomoleculen met optische nano-antennes te kunnen volgen in de tijd, begonnen we een gedetailleerd onderzoek naar de elektronenoverdracht in het metaal-eiwit azurine. Tijdens dit onderzoek bleek dat afzonderlijke eiwitmoleculen dynamische heterogeniteit laten zien, d.w.z. fluctuaties van hun reactiesnelheden, die we toeschrijven aan overgangen tussen conformationele subtoestanden van het eiwit. Dit proces wordt geacht fundamenteel te zijn voor de functie van eiwitten en wordt besproken in het laatste hoofdstuk.


Met Fluorescentie Correlatie Spectroscopie (FCS) kan men de signaalfluctuaties waarnemen die het gevolg zijn van diffusie van fluorescente moleculen in een vloeistof. Diffusiekarakteristieken en reactiekinetiek van biomoleculen kunnen worden afgeleid uit de verkregen correlatiefuncties Voor conventionele FCS metingen waarbij gebruik gemaakt wordt van een diffractie gelimiteerd optisch volume van femtoliters, moet de concentratie
van fluorescerende moleculen in het pico- tot nanomolaire bereik liggen. Echter, de meerderheid van fysiologische reacties met eiwitten en enzymen vindt plaats bij concentaties van micro- tot millimolair.  \textbf{Hoofdstuk 2} beschrijft het gebruik van een gouden nano-staafje als een optisch middel  om de diffractielimiet te overwinnen en FCS bij micromolaire concentraties uit te voeren. De fluorofoor ATTO647N met een kwantum-opbrengst van \SI{70}{\percent} werd opgenomen in een dubbele lipidelaag op een glassubstraat. Omdat ATTO647N vrij diffundeerde in de lipidelaag, kon het toegang krijgen tot de ‘hot spots’ van de nano-staafjes. Een maximale fluorescentieverbetering met een factor vijf werd waargenomen, hoewel berekeningen aangaven dat het nano-staafje de fluorescentie van deze kleurstof met twee ordes van grootte zou kunnen verbeteren op de optimale locatie binnen de hot spot. We schrijven de vrij kleine verbetering toe aan de beperking van de beweging van de kleurstof in de lipidelaag, die toegang tot de heetste (hoogste elektrische veld) plek dicht bij de tip van het nano-staafje verhindert. De autocorrelatiefunctie van de intensiteit omvatte twee componenten afkomstig van diffusie in de laserfocus ( lage fluorescentie-intensiteit) en diffusie in het elektrische ’near field’  van het nano-staafje (hoge fluorescentie-intensiteit). Het volume van het ‘near field’ kon worden berekend uit de diffusietijd en de diffusiecoëfficiënt in de dubbele lipidelaag. De lage intensiteit en hoge achtergrond van  het signaal afkomstig van het diffractie-gelimiteerde volume verlagen het contrast van de ‘near field’ component. Echter, de levensduur van het versterkte signaal is veel korter dan die van het niet-versterkte signaal. We hebben deze eigenschap gebruikt om het signaal van de moleculen in de ‘hot spot’ te scheiden van de rest van het signaal. Door selectie van fotonen op hun levensduur verbetert het contrast tussen de twee componenten aanzienlijk. Hoe hoger het correlatiecontrast, des te hoger is de concentratie van moleculen die kan worden gebruikt in FCS-metingen. Als test is azurine gebruikt dat was gelabeld met ATTO655 en verankerd in de lipidelaag. Diffusie in zowel het ’far field’  als het ’near field’ werd vertraagd met een factor vijf vanwege het omvangrijke anker van het eiwit in de dubbele laag. Het grootste voordeel van het gebruik van een dubbellaag was de volledige verwijdering van niet-specifieke interacties tussen het gelabelde eiwit en het substraat.
 


Vrij diffunderende kleurstoffen in de lipidelaag rond een gouden nano-staafje leiden tot een exponentiële verdeling van de versterkingsfactoren in het ’near field’. Bovendien zijn de emissie- en uitdovingssnelheden van fluoroforen ook nog sterk afhankelijk van hun positie en oriëntatie ten opzichte van het nano-staafje. Een dergelijke variatie in excitatie- en emissiesnelheden resulteert niet alleen in ongewenste heterogeniteit, maar kan ook de fotochemie van de fluoroforen veranderen. \textbf{Hoofdstuk 3} beschrijft  een methode gebaseerd op de kortstondige binding door DNA om deze heterogeniteit te verminderen. Door het meten van de fluorescentie van een molecuul dat steeds op dezelfde plek aan het uiteinde van het gouden nano-staafje is gebonden, kunnen we een groot aantal fluoroforen bestuderen met dezelfde excitatie- en emissiewaarden. Een klein aantal afzonderlijke DNA-strengen (koppelingsstrengen) werden chemisch gebonden aan het oppervlak van het nano-staafje terwijl de rest van het oppervlak van het nano-staafje werd afgedekt met polyethyleenglycol (PEG). Complementair enkelstrengs DNA (10 basenparen)) gelabeld met Cy5 werden geïnjecteerd rond het nano-staafje om te hybridiseren met de koppelingsstrengen. De losse strengen binden tijdelijk aan de koppelingsstrengen waardoor verhoogde fluoresentie-intensiteit wordt verkregen met een typische duur van een paar seconden. Het gemiddelde aantal koppelingsstrengen kon tot één per nano-staafje worden gereguleerd door de verhouding tussen de koppelingsstrengen en de passiverende liganden aan te passen. Variatie in de positie van de koppelingsstreng leidde tot variatie van versterkingsfactoren tussen verschillende nano-staafjes. De versterkingsfactoren zijn exponentieel verdeeld met een gemiddelde waarde van 25. Als een visualisatie-streng bij de koppelplek uiteindelijk afkoppelt en wordt vervangen door een nieuwe visualisatie-streng, dan wordt dezelfde versterkingsfactor waargenomen. Wanneer een nano-staafje met een goede versterkingsfactor werd gevonden, dan kon deze in detail worden gekarakteriseerd en langdurig gebruikt worden om metingen aan één enkel molecuul uit te voeren.


Fluorescentieverbetering wordt vaak geschat door integratie van de fluorescentie als functie van de tijd met een bepaalde integratietijd. De standaardmanier om de versterkingsfactor te schatten is door de ‘burst’ met de hoogste intensiteit te selecteren ('cherry-picking'). De integratietijd is echter een willekeurige parameter en er is geen garantie dat verschillende tijdschalen leiden tot dezelfde versterkingsfactor. In \textbf{hoofdstuk 4} gebruiken wij  de tijd  tussen twee opeenvolgende fotonen  om informatie te verkrijgen zonder een extra parameter te introduceren. We relateren de inter-foton distributie aan de ruimtelijke intensiteitsverdeling zoals die wordt bemonsterd door diffunderende moleculen in de limiet van langzame diffusie. Ons model is getest voor het eenvoudige geval van een emitter die schakelt tussen twee intensiteitsniveaus, zoals verkregen in kortstondige bindingsexperimenten. De inter-foton verdelingen van diffunderende fluoroforen in lipidelagen komen overeen met het model in de limiet van hoge concentraties, maar wijken daarvan af bij lage concentraties. Deze afwijking kan te wijten zijn aan achtergrondfotonen of aan bleking van de kleurstof.


Het volgende doel van het project was om de fluorescentieverbetering te combineren met een gouden nano-staafje met een redox-actief eiwit (koper-azurine), voorzien van een fluorofoor. Het gefunctionaliseerde gouden nano-staafje kan in de toekomst worden gebruikt als een sensor om redox-potentialen te meten in levende cellen. Fluorescentieverbetering door het staafje zou het contrast kunnen verbeteren tegenover cel-autofluorescentie, en daarmee de kwaliteit van de waarneming verbeteren. Echter, wij realiseerden ons dat de elektronenoverdrachtseigenschappen van azurine niet goed genoeg waren gekarakteriseerd op het niveau van één enkel molecuul. \textbf{Hoofdstuk 5} rapporteert een in vitro studie van elektronenoverdracht in enkele azurine-moleculen, onder een elektrochemisch gecontroleerde potentiaal. Koper-azurine gelabeld met ATTO655 vertoonde duidelijke verschillen in fluorescentie-intensiteit in de geoxideerde en gereduceerde toestand van het koper. De ’midpoint’ potentialen van een aantal azurine-moleculen werden berekend uit de verhouding van de lengtes van de heldere en donkere periodes met behulp van de Nernst-vergelijking. De gemeten midpointpotentiaal bedroeg 5mV met een verdeling  van \SI{35}{\mV}, wat de kleinste waarde is die tot nog toe  gerapporteerd is in de
literatuur. Zo'n nauwe verdeling van ’midpoint’ potentialen wijst op een minimale interactie van het azurine met het substraat. Echter, de oorsprong van de resterende breedte, en de aard van deze verdeling naar ruimte en tijd was onduidelijk. Om deze vragen te beantwoorden hebben wij enkele azurine-moleculen gedurende langere tijd onderzocht. De histogrammen van de heldere en donkere periodes in de fluorescentie vertoonden een ‘stretched-exponential’ verdeling, een aanwijzing voor variatie in de tijd van de elektron overdracht. We schrijven dergelijke variaties toe aan variaties in de conformatie  van azurine, bekend als conformationele subtoestanden. De autocorrelatie van de heldere en donkere perioden had een vervaltijd van tientallen seconden, hetgeen de gemiddelde tijd aangeeft die azurine in een bepaalde conformatie doorbrengt. Dat er geen twee individuele azurine-moleculen werden gevonden met identieke correlaties, wijst op  een zeer groot aantal conformationele subtoestanden.