\documentclass{dissertation}

%% Turn off page numbering for the propositions and make the margins on both
%% sides equal and symmetrical.
\geometry{twoside=false}
\pagestyle{empty}

\begin{document}

%% Specify the title and author of the thesis. This information will be used on
%% both the English and Dutch side and in the metadata of the final PDF.
\title[]{Fluorescence of Single Copper Proteins: Dynamic Disorder and Enhancement by a Gold Nanorod}
\author{Biswajit}{Pradhan}

\begin{center}

{\Large\titlefont\bfseries Propositions}

% \bigskip

accompanying the dissertation

% \bigskip

%% Print the title.
{\makeatletter
\titlestyle\bfseries\large\@title
\makeatother}

%% Print the optional subtitle.
{\makeatletter
\ifx\@subtitle\undefined\else
    \titlefont\titleshape\@subtitle
\fi
\makeatother}

% \bigskip

by

% \bigskip

%% Print the full name of the author.
\makeatletter
{\large\titlefont\bfseries\@firstname\ {\titleshape\@lastname}}
\makeatother

\end{center}

% \bigskip
% \bigskip
\newcommand{\refp}[1]{\\{\it \footnotesize #1}.}
\begin{enumerate}
	\item The shortened lifetimes observed in fluorescence-enhancements experiments using plasmonics are primarily due to nonradiative losses to the metal surfaces.
	\refp{Chapter 2 and 3 of this thesis}
	
	\item Kinetic studies of biomolecular processes by plasmonic sensing require partial or full immobilization. 
	\refp{Chapter 2 and 3 of this thesis}
	
	\item Interphoton time-delay is an overlooked parameter in single-molecule fluorescence spectroscopy.
	\refp{Chapter 4 of this thesis}
	
	\item Histograms of bright and dark times in the time trace of the fluorescence of a single molecule can hide rare events.
	\refp{Chapter 5 of this thesis}
	
	\item Immobilization can alter catalytic activities of enzymes.
	\refp{Francesco Secundo, Chem. Soc. Rev. \textbf{42} , 6250-6261(2013)}
	
	\item The magnitude of fluorescence enhancement by plasmonics is only informative when the quantum yield of the unenhanced fluorophore is specified.
	\refp{Puchkova et al., Nano Lett. \textbf{15}, 12(2015);
		  Acuna et al., Science \textbf{338}, 506(2012);
		  Yuan et al., Angewandte Chemie \textbf{52}, 1217-1221(2013)}
	
	% \item Angular variations could be the major reason for variation in the electron transfer rate in single proteins rather than distance between donor and acceptor, contrary to Yang et al.'s report.	
	\item Contrary to what is reported by Yang et al., the relative orientation of donor and acceptor might have the major contribution to the variation in the electron-transfer rates in single proteins.
	\refp{Yang et al., Science \textbf{302}, 262-266 (2003)}
	
	\item The term `protein dynamics' does not give much insight into the detailed fluctuations as long as length and time scales are not specified.
	\refp{Kern et al., Nature \textbf{450}, 964–972 (2007)}
	
	% \item To some extent, humans are responsible for the frequently occurring current natural disasters.
	\item `Data available on request' is equivalent to `data inaccessible'.
	% \item The Netherlands does not have mountains.

	\item Dreams are no different than realities.
\end{enumerate}

\end{document}

