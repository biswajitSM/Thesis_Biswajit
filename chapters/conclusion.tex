\chapter*{Conclusion and Outlook}
\label{ch:conclusion}
\markboth{Conclusion}{}
\addcontentsline{toc}{chapter}{Conclusion and Outlook}

% \paragraph*{General conclusion}
This thesis is focused on both the methodology and applications of single-molecule fluorescence. The first three chapters deal with fluorescence enhancement by single gold nanorods which allow single-molecule investigations of optical absorbers with low quantum yield. In the second part, we study electron transfer in single metalloproteins. Here we conclude by summarizing the important outputs of the thesis, possible applications and prospects for future experiments.

\paragraph*{Enhanced FCS}
Enzymatic reactions occur in micromolar to millimolar concentrations. We demonstrated a plasmonically enhanced fluorescence method to observe single molecules at a higher density of molecules on a lipid bilayer. A resonant gold nanorod squeezed the electromagnetic energy into a thousand times smaller volume than the diffraction limit of an optical microscope. We functionalized the bilayer with a high quantum yield dye to characterize both near-field and far-field of the nanorod antenna from the same measurement. The intensity fluctuations due to the diffusion of the emitters were characterized by fluorescence correlation spectroscopy. Two diffusion times were obtained from the correlation curve corresponding to the near field and far field. The near-field volume could directly be estimated from the diffusion times and the known diffusion coefficient in the lipid bilayer. The lifetime of the enhanced fluorescence was shortened by the nanoantenna. The correlation amplitude improved by more than an order of magnitude by correlating photons with ten times shorter lifetimes than the un-enhanced fluorescence. We also functionalized the bilayer with labeled protein and observed enhanced correlation without any physical and chemical harm to the biomolecules.

Our experiment was a preliminary demonstration of enhanced fluorescence correlation spectroscopy and characterization of near-field. A more application-oriented approach would be to use a dye with low quantum yield. Two immediate benefits with weak emitters are: i) fluorescence can be enhanced by three to four orders of magnitude providing high contrast against the far-field, ii) low fluorescence background from the far-field will cause a mild reduction on the correlation amplitude. Further more, nanorods with surface plasmon at longer wavelength can be used for better fluorescence enhancement and lower natural fluorescence background inside a cell. In addition to its biomimicking function, a lipid bilayer also provides a ground for anchoring biomolecules. Enzymes and proteins can slowly diffuse in the near-field while we observe their biological activity. For cellular membranes, the diffusion in the near-filed can reveal the patchy domains at high resolution.

\paragraph*{Transient Binding}
In the previous chapter, we obtained distributed enhancement factors for freely diffusing dyes around a gold nanorod. Strong inhomogeneity of the electric field around plasmonic nano-antenna leads to a large variation in enhancement. We demonstrated a DNA-based transient binding method to observe fluorescence from a fixed point on the gold nanorod. Many single molecules were observed, one at a time with the same enhancement factor. The long persistence length of DNA allowed us to fix the position of the emitter. We controlled the average number of binding sites down to one per gold nanorod with proper passivation of the gold surface.

The rich and robust chemistry of DNA allows us to control distance and binding energy more efficiently. One of the challenge in plasmonic enhancement is to re-construct a three dimensional map of fluorescence enhancement (or electric field) around a nano-antenna. With high spatial (0.35~nm resolution per DNA base) control of emitters by DNA, an experiment can be envisioned to transiently bind different lengths of imager (emitter with a short strand DNA) and construct a 3D map using the principles of super-resolution. As DNA-technology is bio-inspired, plasmonics with DNA can be a natural choice to perform enhancement experiments inside living cells.

\paragraph*{Bin-free analysis}
Intensity or number of photons per unit time (binning time) is a typical quantity used to characterize fluorescence. Such binned traces give accurate dynamics if the binning time is shorter than the underlying time of intensity fluctuation. In addition, a high signal to noise ratio is required to get better accuracy of the dynamic parameters. What if the underlying kinetics is unknown? What binning time should be chosen? With different binning times, we observed different enhancement factors for the diffusion of dyes in the near field of a gold nanorod. Accurate estimation of the enhancement factor would help in characterizing the plasmonic effect on the fluorophores. In this chapter, we proposed a binning-free method with the delay time between two consecutive photons. Numerical solutions were obtained for the distribution of inter-photon times for the static and dynamic intensities arising from different numbers of molecules and different excitation-detection profiles. Transient binding on a gold nanorod and diffusion in a bilayer were satisfactorily predicted from their inter-photon times. For free diffusion in the bilayer, accuracy of our model proved to be better at higher number of molecules. Similar analysis is being carried out for enhanced fluorescence. We hope to devise a more general binning-free method to characterize fluorescence dynamics.

\paragraph*{Electron transfer in azurin}
Do enzymes and proteins fluctuate in their structure? In the final chapter, we monitored electron transfer rates of single azurins as a function of time. Dynamic heterogeneity was observed from the analysis of time traces, histograms of lifetimes of bright and dark states, and correlation of bright and dark times. Non-exponential (more than two) distributions of bright and dark times indicated distributions in the electron transfer rates. As the electron transfer rates are monitored in steady state, their variation in activity were attributed to different conformations of azurin. The decay in the correlation of dark and bright times indicated the correlated variation in their electron transfer rates. In simple terms, a longer bright time was followed by a longer bright time and a shorter bright time was followed by a shorter bright time. The decay time in the correlation gives the duration for which an azurin retains a certain conformation. No two azurin showed same values of dynamic correlation indicating the large number of conformational substates a protein can possess. We suppose such changes come from the movement of domains in a protein where weak interactions like hydrogen bonding, hydrophobic interactions give a compact form to a long peptide chain. 

Though we are reasonably confident about the presence of substates in azurin, the reason why proteins change their conformations is an open question. No computational or theoretical model to the best of our knowledge has predicted dynamic correlations in the time scale of seconds. Azurin is comparably a small and stable protein. From the presence of heterogeneity in azurin, it is tempting to assume a general presence of conformational substates in other proteins.

We performed the electron transfer experiment \textit{in vitro} on a non-interacting functionalized glass substrate. To convince ourselves about the generality of dynamic heterogeneity, single-enzyme experiments can be performed in biological-mimicking structures like lipid vesicles where interaction from external sources can be minimized.