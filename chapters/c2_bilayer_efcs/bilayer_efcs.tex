\chapter{Gold-Nanorod-Enhanced FCS of Fluorophores with High Quantum Yield in Lipid Bilayers}
\label{chapter:EFCS}

\blfootnote{This chapter have been published in Journal Physical Chemistry C 2016, 120, 25996−26003.}

\begin{abstract}
	Plasmonic fluorescence enhancement is used to study fluorescence correlation spectroscopy (FCS) at higher concentrations than in regular diffraction-limited FCS experiments. Previous studies suffered from sticking to the substrate and were performed mainly with poorly emitting dyes.  A lipid bilayer forms a passivating surface preventing sticking of the dye or the protein and allows for specific anchoring of probe molecules. For dyes with high quantum yields, the fluorescence background of unenhanced molecules is high, and the fluorescence enhancement is weak, less than about 10. Nonetheless, we show that FCS is possible at micromolar concentrations of the probe molecule. Enhanced FCS is recorded by selecting signals on the basis of their shortened lifetime. This selection enhances the contrast of the correlation by more than an order of magnitude. The lipid bilayer can be used to anchor biomolecules and perform enhanced FCS, as we show for a dye-labeled protein.
\end{abstract}

\section{Introduction}
Fluorescence-based single-molecule detection helps exploring the structure and dynamics of complex biological matter.\cite{moerner1999illuminating} Single-molecule signals can reveal a transient state during a chemical reaction, or report on the kinetics of processes as a function of position. Broadly there are two ways of studying single molecules: i) by immobilizing the molecule on a background-free matrix or surface, ii) by dissolving the molecules in a fluid and measuring the signal fluctuation by fluorescence correlation spectroscopy (FCS).\cite{Magde1972} Both techniques require the molecule to possess a high quantum yield and good photostability. FCS studies are limited to concentrations in the pico- to nano-molar range, in view of the diffraction-limited detection volume of a few femtoliters (fL). As many biological reactions occur in the micromolar range\cite{craighead2006future}, smaller detection volumes are desirable to study these reactions by FCS.

\section{Method}
\section{Result and discussion}
\subsection{Fluorescence enhancement}
\subsection{Lifetime as additional identifier of enhanced signal}
\subsection{FCS and contrast improvement}
\subsection{EFCS with biomolecules}
\section{conclusion}
\references{chapter/c2_bilayer_efcs/efcs}